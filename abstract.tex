
\begin{abstract}
\textbf{TODO: This needs some work. \\}
The Heterogeneous Computing model has different processors for throughput-intensive and latency-sensitive tasks.
Latency-optimized CPUs use complex branch prediction, speculative execution, register remaining, etc.
This requires a lot of silicon surface and thus a lot of energy.
Throughput-intensive systems reduce these complexities and work at minimal energy per bit.
CPU-systems cannot go faster anymore.
There are many ideas for configuring systems that work more efficiently under heterogeneous workloads.
In debate are the CPU and accelerator architecture, the types of inter-/intra-node interconnects, and the heterogeneous vs self-hosted model.
One of the aproaches is using an heterogeneous system formed by a combination of CPUs and GPUs.
Nvidia is betting strongly on these systems, building their GPUs not only for graphics but for High Performance Computing.
The advantage of GPUs for computation is achieved by dedicating more transistors to arithmetic logics and less to control.
In this document I go through the most important hardware updates in Pascal emphasizing NVLink.
NVLink is important because it breaks some of the bottlenecks appearing at GPU-to-GPU communication.
Also a comparison of Pascal to previous architectures is presented.
At the end there is a description of an attempt to create a general hardware model for GPGPU.
 % Expose algorithmic limitations (focusing on deep learning) of the previous architectures and where improvements can be expected in the newer platform.
\end{abstract}
